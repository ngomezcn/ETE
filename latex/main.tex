\documentclass[11pt]{article}
\usepackage{amsmath}

\usepackage[margin=1in]{geometry} % Puedes ajustar el valor del margen según tus necesidades

\begin{document}

\section*{Cálculo de Una Variable}

 \item \textbf{CUVprobl-Derivades.pdf}
    \begin{enumerate}
        \item \textbf{P.1} Calculeu la primera derivada de les funcions següents
            \begin{align*}
                (a) &\quad y = \arctan(\sqrt{x^2+1})   \\
                (b) &\quad y = x^{2x}2^x  \\
            \end{align*}

        \item \textbf{P.2} Determineu la derivada de y = f(x) donada com a funció implícita per les equacions següents:
            \begin{align*}
            (a) &\quad \sqrt{x} + \sqrt{y} = \sqrt{a}  \\
            (b) &\quad x \, \sin(y) - y^2 = 0  \\
            \end{align*}

        \item \textbf{P.2} Determineu els valors de a, b i c per tal que la funció:
            \begin{align*}
                (a) &\quad f(z) = \left\{ \begin{array}{rcl}
                            x^3 & \mbox{si} & x < -1 \\
                            ax^2 + bx + c & \mbox{si} & -1 > x
                                    \end{array}\right.  \\
            \end{align*}
        sigui contínua amb derivades primeres i segones contínues \(\forall x  \in \mathbb{R}\).
        Determineu quina discontinuïtat té la derivada tercera en \(x = -1\).

        \item \textbf{P.4} Calculeu els següents límits aplicant la regla de l'Hôpital:
            \begin{align*}
            (a) &\quad \lim_{{x \to 0}} \left(\cot{x -\frac{1}{x}}\right)  \\
            (b) &\quad \lim_{{x \to \infty}} \frac{\sin(\frac{1}{x})}{e^{-x}}  \\
            (c) &\quad \lim_{{x \to \pi/2}}   \left[ \left(x - \frac{\pi}{2} \right) \tan(x) \right]  \\
            \end{align*}




        \item

    \item \end{enumerate}






\nocite{cite_key}
\end{document}
